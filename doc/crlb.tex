\documentclass[12pt]{article}

\usepackage{amsmath}
\usepackage{amssymb}
\usepackage{amsfonts}
\usepackage{amsthm}
\usepackage{mathrsfs}
\usepackage{colonequals}
\usepackage{subfig}
\usepackage[shortlabels]{enumitem}
\usepackage[margin=1in]{geometry}
\usepackage{caption}
\usepackage{color}
\usepackage{listings}
\usepackage{bm}

\theoremstyle{plain}
\newtheorem{thm}{Theorem}

\theoremstyle{definition}
\newtheorem{prob}{Problem}

\usepackage{color}
\usepackage{graphicx}
%\graphicspath{./figs/}

\newcommand{\TODO}{{\color{red}TODO}}
\newcommand{\CC}{\mathbb{C}} % complex numbers
\newcommand{\RR}{\mathbb{R}} % real numbers
\DeclareMathOperator{\rank}{rank}
\newcommand{\norm}[1]{\left\lVert #1 \right\rVert}
\newcommand{\ip}[2]{\langle #1, #2 \rangle}
\newcommand{\ones}{\mathbf{1}}
\newcommand{\nth}[1]{#1^{\mathrm{th}}}
\newcommand{\TR}{\mathrm{TR}}
\newcommand{\TE}{\mathrm{TE}}
\renewcommand{\vec}[1]{\mathbf{#1}}
\newcommand{\veca}[1]{\boldsymbol{#1}}

\title{Optimal Experiment Design}
\author{Jon Tamir}
\begin{document}
\maketitle


\section{Signal Equation}
The signal equation is
\begin{align}
  S_{ij}\left(M_0, T_1, T_2\right) &= M_0 f_i(T_1, T_2) \frac{1 - E_j(T_1)}{1 - E_j(T_1)f_T(T_1,T_2)},
  \label{eqn:signaleq}
\end{align}
where $M_0$ is the initial longitudinal magnetization and $T_1$ and $T_2$ are relaxation parameters.
The EPG function is
\begin{align}
  f_i(T_1, T_2) &= \mathrm{EPG}\left( \theta_1^i, T_1, T_2, T_s \right),
  \label{eqn:epgfun}
\end{align}
where $\theta_1^i$ indicates the first $i$ refocusing pulses and $T_s$ is the echo spacing. The recovery function is
\begin{align}
  E_j(T_1) &= e^{-\frac{(\TR_j - (T+1)\times T_s)}{T_1}},
  \label{eqn:recoveryfun}
\end{align}
where $\TR_j$ is the $\nth{j}$ repetition time.

The Fisher Information Matrix (FIM) consists of the following six terms:

\begin{align}
  \vec{I} = \begin{bmatrix}
  \frac{2}{\sigma^2} \sum_{ij}\left( \frac{\partial S_{ij}}{\partial T_1} \frac{\partial S_{ij}}{\partial T_1} \right) & 
  \frac{2}{\sigma^2} \sum_{ij}\left( \frac{\partial S_{ij}}{\partial T_1} \frac{\partial S_{ij}}{\partial T_2} \right) &
  \frac{2}{\sigma^2} \sum_{ij}\left(\frac{\partial S_{ij}}{\partial T_1} \frac{\partial S_{ij}}{\partial M_0} \right) \\
  \cdot & \frac{2}{\sigma^2} \sum_{ij}\left( \frac{\partial S_{ij}}{\partial T_2} \frac{\partial S_{ij}}{\partial T_2} \right) &
  \frac{2}{\sigma^2} \sum_{ij}\left( \frac{\partial S_{ij}}{\partial T_2} \frac{\partial S_{ij}}{\partial M_0} \right) \\
  \cdot & \cdot &
  \frac{2}{\sigma^2}\sum_{ij}\left(\frac{\partial S_{ij}}{\partial M_0} \frac{\partial S_{ij}}{\partial M_0} \right) &
\end{bmatrix}.
  \label{eqn:FIM}
\end{align}

For ease of notation, a function $f(x,y,z)$ will be referred to as $f(x)$ when it is understood that the other parameter
are constants in the expression. Also, the partial derivative of a function $f(x,y,z)$ w.r.t.\ parameter $x$ will be denoted by $f'(x)$.
We have
\begin{align}
  S_{ij}'(M_0) &= f_i \frac{1 - E_j}{1 - E_jf_T},
  \label{eqn:dM0} \\
  \begin{split}
  S_{ij}'(T_1) &= M_0\left( 1 - E_j(T_1) \right)\frac{\left(f_i'(T_1)\left(1 - E_j(T_1) \right) - f_i(T_1)E_j'(T_1)\right)}
  {\left(1 - f_TE_j\right)^2} \times \\
  & \frac{\left(1 - f_T(T_1)E_j(T_1)\right) + f_i(T_1)\left(f_T'(T_1)E_j(T_1) + f_T(T_1)E_j'(T_1)\right)}
  {\left(1 - f_TE_j\right)^2},
  \label{eqn:dT1}
  \end{split} \\
  S_{ij}'(T_2) &= M_0\left( 1 - E_j \right)\frac{\left(f_i'(T_2)\left(1 - f_T(T_2)E_j \right) + f_i(T_2)f_T'(T_2)E_j\right)}
  {\left(1 - f_TE_j\right)^2}.
  \label{eqn:dT2}
\end{align}

The inverse of the FIM is a lower bound on the covariance matrix of an unbiased estimator of the parameters $\veca{\theta}=(T_1, T_2, M_r)$.
Thus it reasons to minimize some metric of sub-matrix of $\vec I$.

Group the FIM into
\begin{align}
  \vec I &= \begin{bmatrix}
    \vec I_{11} & \vec I_{12} \\ \vec I_{21} & \vec I_{22},
  \end{bmatrix}
  \label{eqn:FIMd}
\end{align}
where $\vec I_{11}$ represents the $2\times 2$ submatrix. Then the Schur complement of $\vec I$ w.r.t.\ $\vec I_{22}$ is

\end{document}

